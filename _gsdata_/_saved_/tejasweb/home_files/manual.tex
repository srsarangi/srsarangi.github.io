\documentclass[margin=0.1in]{article}

\usepackage{url}

\begin{document}

\title{Setting Up {\em Tejas}}
\date{}
\maketitle


\subsection*{Getting Started}

\begin{enumerate}

\item
Place the installation script {\em install\_tejas.py} in the folder where you wish to place the source code of Tejas. A copy of the script can be found at \url{http://www.cse.iitd.ernet.in/~srsarangi/tejas/home_files/install_tejas.py}.

\item
Run the script : {\em python install\_tejas.py}.

\item
The script first prints the list of required packages. Make sure you have the same installed.

\item
You will be prompted for your installation method preference -- `1' to 
install from a downloaded tar-ball (tar-balls available at \url{http://www.cse.iitd.ernet.in/~srsarangi/tejas/install.html})
, or `2' to clone a mercurial repository. If you decide to clone,
use ``guest" as the username and ``guest1" as the password.

\item
Tejas uses Intel PIN. Please download version 62732 from \url{http://software.intel.com/en-us/articles/pintool-downloads}.
Please read the licence at \url{http://software.intel.com/sites/landingpage/pintool/extlicense.txt} before downloading.
Provide the absolute path to the downloaded tar-ball at the prompt.

\item
The script then builds Tejas, configures it, and simulates a simple ``Hello World" program
to help verify successful installation. Please check if the output file is generated (the path of the output file is printed by the script).


\end{enumerate}


\subsection*{Running Simulations}

\begin{itemize}

\item
The configurations used by the simulator (for example, size of the L1 cache) are mentioned in a configuration file -- a sample file is given in {\em tejas-dir/src/simulator/config/config.xml}. Create copies of this file and make changes as required.

\item
Run the following commands in {\em tejas-dir}:
\begin{itemize}
\item ant clean
\item ant
\item ant make-jar -- this creates a jar file {\em tejas-dir/jars/tejas.jar}.

To change the jar filename, change line 5 of {\em tejas-dir/build.xml} accordingly.

\item java -jar {\em jar-file} {\em config-file} {\em result-file} {\em benchmark}.

The outcome of the simulation is written to {\em result-file}; {\em benchmark} is the x86 executable that is to be simulated. Note : always use absolute pathnames for specifying the arguments.
\end{itemize}

\item
It is recommended to use the eclipse IDE to work with Tejas. Eclipse can be freely downloaded. Depending on the version of eclipse,

\begin{itemize}
\item
In File$\rightarrow$New$\rightarrow$Java Project, create a project from existing source, namely, {\em tejas-dir}, OR,
\item
Go through File$\rightarrow$Import$\rightarrow$General$\rightarrow$Existing projects into workspace, and use {\em tejas-dir} as the root directory.
\end{itemize}

\end{itemize}


\subsection*{General Instructions}

\begin{itemize}

\item
Please use absolute paths always.

\item
The {\em pin-dir} used in {\em tejas-dir/src/emulator/pin/makefile.gnu.config} and the configuration file must be the same.

\end{itemize}



\subsection*{Trouble Shooting the set-up}

\begin{itemize}

\item
If the error message displayed is ``pin.h not found",\\
check the {\em tejas-dir/src/emulator/pin/makefile.gnu.config} file if {\em PIN\_KIT} is set correctly to {\em pin-dir}.

\item
If the error message displayed is ``jni.h not found", locate where Java is installed on your system, and accordingly update JNINCLUDE in the file {\em tejas-dir/src/emulator/pin/makefile}.

\item
If the error message is of the form :\\
\\
E:Attach to pid 27344 failed.\\
E:  The Operating System configuration prevents Pin from using the default (parent) injection mode.\\
E:  To resolve this, either execute the following (as root):\\
E:  \$ echo 0 $>$ /proc/sys/kernel/yama/ptrace\_scope\\
E:  Or use the "-injection child" option.\\
E:  For more information, regarding child injection, see Injection section in the Pin User Manual.\\
\\
\\
run the command ``echo 0 $>$ /proc/sys/kernel/yama/ptrace\_scope" as the root user.


\end{itemize}

\subsection*{Updating Tejas}
To update the current version of Tejas,
\begin{itemize}
\item
Set up the current source as a mercurial repository (required for source obtained through the tar-ball method)
\item
Run the command ``hg pull http://www.cse.iitd.ac.in/srishtihg/hg/distrib/Tejas''. The username is ``guest" and password is ``guest1".
\item
Run the command ``hg update". Resolve merge conflicts, if any.
\item
See \url{http://mercurial.selenic.com/wiki/Tutorial} and \url{http://hgbook.red-bean.com/} for help regarding mercurial.
\end{itemize}

\end{document}
