\documentclass[a4paper]{article}
\usepackage[a4paper,top=2cm,bottom=2.5cm,left=1.5cm,right=1.5cm,marginparwidth=1.75cm]{geometry}
%% Language and font encodings
\usepackage[english]{babel}
\usepackage[utf8x]{inputenc}
\usepackage{listings}

%% Sets page size and margins

\usepackage{float}
%% Useful packages
\usepackage{amsmath}
\usepackage[colorinlistoftodos]{todonotes}
\usepackage[colorlinks=true, allcolors=blue]{hyperref}
\usepackage{listings}
\usepackage{url}
\usepackage{graphicx}
\graphicspath{ {./images/} }
% \DeclareGraphicsExtensions{.pdf,.jpg,.png}

%% defined colors
\definecolor{Blue}{rgb}{0,0,0.5}
\definecolor{Green}{rgb}{0,0.75,0.0}
\definecolor{LightGray}{rgb}{0.6,0.6,0.6}
\definecolor{DarkGray}{rgb}{0.3,0.3,0.3}

\title{\huge{Tejas-Java: Java Architecural Simulator}}
\author{Indian Institute of Technology, New Delhi\\
http://www.cse.iitd.ac.in/tejas/}
\date{}

\begin{document}
\maketitle

\section{Introduction}
Jikesrvm is an open-source research virtual machine for Java bytecodes. It runs on IA-32
Linux and PowerPC64 Linux platforms. This manual describes how to build, run and collect
the Java program traces by Jikesrvm on IA-32 and x86-64 Linux. x86-64 is not currently
supported but Jikesrvm can be built for x86-64 by using 32-bit addressing. This manual also
describes how to run these traces on Tejas simulator and find the heap memory usage of only
some classes of Java programs.

\section{Getting the Code}

\section{Building JikesRVM}
We are using JikesRVM 3.1.2 which can be downloded from the official Git repository or from here \url{http://sourceforge.net/projects/jikesrvm/files/jikesrvm/3.1.2/}.
\subsection{Dependencies}
\begin{enumerate}
\item \textbf{Java 6}: 3.1.2 version of JikesRVM compiles using Java 6. Building it with +6 versions of Java fails because of langaugre changes. We tested this using openjdk-6.
\item \textbf{Ant}: Requried for build script
\item \textbf{gcc}: Compilaton
\item \textbf{gcc-multilib}: This is required if you are building on a 64 bit machine ( which most probably you are..).
\item \textbf{g++-multilib}: Same as above
\item \textbf{Bison}: Compilation
\item \textbf{Perl}: Build Scripts
\end{enumerate}

\subsection{Proxy Setup}
During the build process, certain jar files are downloaded from the internet. Make sure you have a working Internet connection. If you are behind a proxy, then you need to set some environment variable to make it work.
\begin{itemize}
\item \textit{Ant Proxy:}
\begin{verbatim}
export ANT\_OPTS="-Dhttp.proxyHost=<host here>
-Dhttps.proxyHost=<host here> -Dhttps.proxyPort=<port here>
-Dhttp.proxyPort=<port here> -Dftp.proxyHost=<host here>
-Dftp.proxyPort=<port here>"

export http_proxy=http://<host here>:<port here>
export https_proxy=https://<host here>:<port here>
export ftp_proxy=ftp://<host here>:<port here>
\end{verbatim}
These lines can be added to \textit{~/.bashrc} file, if you are using Ubuntu.
\end{itemize}

\subsection{Configure JikeRVM}
You may get an error java not found if you start building Jikesrvm directly, therefore modify \textit{bin/buildit.base\_config} file and set these variables appropriately\\
For 32-bit and 64 bit systems set 
\begin{verbatim}
global.javahome.ia32-linux = /usr/lib/jvm/openjdk-6-jdk
global.javahome.x86 64-linux = /usr/lib/jvm/openjdk-6-jdk
\end{verbatim}
respectively.
\subsection{Predefined build options}
Jikesrvm can be build with different configurations. Two useful configurations used in this project are:
\begin{itemize}
\item Development:\\
This configurations defines the fully optimized versions of Jikesrvm with debug mode.
It builds and executes slowly. We can use this configuration to collect the traces for
normal execution of Java programs with all the optimizations turned on. This can be
build by:
\begin{verbatim}
$cd jikesrvm
$sudo bin/buildit localhost development
\end{verbatim}
\item FullAdaptive NoGC:\\
This configurations defines no garbage collector versions of Jikesrvm. Since we need
to implement hardware garbage collector. Therefore we require there is no memory
reuse done by software garbage collector. This can be build by:
\begin{verbatim}
$cd jikesrvm
$sudo bin/buildit localhost FullAdaptiveNoGC
\end{verbatim}
Other configurations details can be found in \textit{jikesrvm/build/configs	} directory. One or more
configurations can coexist in the same directory of jikesrvm.
\end{itemize}

\section{Running JikesRVM}
After Jikesrvm build is successful a directory jikesrvm/dist/config name is created. We can check build is successful or not by running a sample Java program. To invoke Jikes virtual machine use rvm script in \textit{jikesrvm/dist/config\_name}. The command for running Java program by Jikes virtual machine:
\begin{verbatim}
jikesrvm/dist/config name/rvm <java class name>
\end{verbatim}
The program should run without crashing Jikesrvm.

\section{Command Line options for JikesRVM}
\begin{itemize}
\item \textit{-X:aos:enable recompilation=false -X:aos:initial compiler=base}\\
This option will turn off all the runtime compiler optimization on the program.
\item \textit{-X:base:mc=true}\\
This option can be use for printing the assembly instruction generated by the baseline
compiler. It can be useful to compare the traces generated by jikes trace.
\item \textit{-X:opt:mc=true}\\
This option can be used for printing the assembly instructions generated by the optimizing
compiler. It prints the assembly instructions only for the classes which could
be optimized by the virtual machine.
\item \textit{-X:gc:ignoreSystemGC=true}\\
This option can be used to ignore the system calls for garbage collection. It can be
useful while running the program with no garbage collection mode.
\item \textit{-X:gc:harnessAll=true}\\
This option can be used to print the stats related to garbage collectors like time spent
in running mutator program and garbage collector.
\item \textit{-help}\\
This option can be used to find the various option.
\end{itemize}

\section{Tejas + JikesRVM}
The instrumented code is used to get the trace of the program currently running, which then can be used along with Tejas to get performance readings.\\
The instruction and memory traces of the java program can be obtained as follows:
\begin{itemize}
\item Write a java program \textit{hello.java}.
\item Get trace file using following commands:
\begin{verbatim}
$javac hello.java
$cp hello.class JIKES_HOME/dist/development-xxxx/
$cd JIKES_HOME/dist/development-xxxx/
$./rvm hello 2>trace file
\end{verbatim}
\textbf{Note:} The trace file generated will have the traces of hello program. By default optimizations are enabled, therefore if you don’t use any options the traces generated will be of optimized code.\\\\
\textbf{Baseline Compiler:} The execution traces of the program without any optimizations can
be collected by:
\begin{verbatim}
$./rvm -X:aos:enable recompilation=false -X:aos:initial compiler=base hello 2>trace file
\end{verbatim}
	
\item \textbf{No Garbage Collector:} The execution traces of the program without garbage collector activity can be collected by:
\begin{verbatim}
$cd JIKES_HOME/dist/FullAdaptiveNoGC_xxxx/
$./rvm -X:gc:ignoreSystemGC=true -X:aos:enable recompilation=false 
-X:aos:initial compiler=base hello 2>trace file
\end{verbatim}
\end{itemize}

\subsection{Multithreaded Programs:}
For multithreaded programs use rvm multithread (rvm modified for multithreaded programs) and build jikesrvm again as described above. The traces generated by the Jikesrvm for multithreaded programs are interleaved. Corresponding to each instruction and memory traces a threadId is also printed, we split these interleaved traces into multiple trace file one for each thread: 
\begin{verbatim}
python splitter.py trace file
\end{verbatim}
trace file is the raw trace file generated by jikesrvm multithread. This generates multiple trace files. Each thread will have one trace file named as its threadId. These trace files should be post processed individually.\\\\
\textbf{Partial Traces:} In case you don’t want the complete trace of a program, you can specify the number of instructions to be traced. For this modify \textit{jikesrvm/rvm/src/org/jikesrvm/runtime/RuntimeEntrypoints.java}. Search for my check function and add these statements at the beginning of function definition.
\begin{verbatim}
VM.inst_count++;
if(VM.inst_count == <noOfInst>) // noOfInst is the integer value of number of 
//instructions to be traced.
VM.sysFail("instruction completed");
\end{verbatim}

At the end of trace files there will be virtual machine crash messages, those statements should be removed before disassembling the traces. 
\begin{verbatim}
cat trace file | head - n -30 >trace file1
\end{verbatim}
For multithreaded programs noOfInst should be sum of all instructions to be traced of each
thread.


\section{Disassemble to x86 Instructions}
The traces generated by Jikesrvm are in a binary format. To run these traces on the architectural simulator Tejas, we require traces in VISA format(instruction set of Tejas). The x86 to VISA translator is already implemented in Tejas. Therefore we need to disassemble these raw traces into x86 instructions. We use the udis86 library for this purpose. libudis86 is a disassembler library for the x86 architecture which decodes a stream of bytes as x86 instructions.
\begin{itemize}
\item Separate the instruction and memory traces
\begin{verbatim}
python conv.py <tracefile>
\end{verbatim}
It generates two files temp file.txt and out trace.txt.\\
\textbf{temp\_file.txt:} has 8 bytes of instructions in hexadecimal format per line which is the input for the udis86 disassembler.\\
\textbf{out\_trace.txt:} has instruction pointer and the memory address accessed by the program which we use later to merge the instruction and memory trace.
\item Disassemble into x86 instructions.\\
Download the libudis86 and untar:
\begin{verbatim}
$./configure
$make
$sudo make install
$compile disas.cpp using $g++ -std=c++11 disas.cpp -o disas.o -ludis86
$./disas.o
\end{verbatim}
\textit{disas.o} reads the instructions bytes from temp file.txt and disassembles it. The x86 instructions are of variable length. Therefore to disassemble it disas.o reads as many
bytes required to form a instruction and ignores the remaining ;ytes in the line. The
next instruction to be disassembled is read from the next line. The disassembled instructions are written to temp\_file1.txt.
\item Merge disassembled instruction traces and memory traces
\begin{verbatim}
$javac conv1.java
$java conv1
\end{verbatim}

It generates fin \textit{trace.txt\_0} which will be in right format to run on Tejas simulator.
\end{itemize}

\subsection{Running on Tejas}
\begin{itemize}
\item Compress trace files using
\begin{verbatim}
gzip -c trace file >trace file 0.gz
\end{verbatim}
Compressed file name should end with x.gz. For single threaded program x should be
0. And for the multithreaded programs x is the thread number.
\item Modify Tejas config file\\
Set the following parameters in Tejas config file:
\begin{itemize}
\item EmulatorType: none
\item CommunicationInterface: file
\end{itemize}
\item Run Tejas
\begin{verbatim}
java -jar <tejas-jar><config-file><output-file><trace file>
\end{verbatim}
\end{itemize}

\section{ Heap memory usage of program}
The dynamic memory used by the program with and without garbage collector can be found
by annotating benchmarks, collecting the traces using Jikesrvm, post processing the traces
and running on Tejas.
The address of the dynamically created objects are visible in the traces when we access the fields of the objects. Therefore to find the memory address we insert the markers in the source code and using this markers we extract the memory addresses of dynamically created objects.

\subsection{Annotating benchmark}
\begin{itemize}
\item Object creation:\\
In order to keep track of the heap memory used by the program, annotate the source
code at places where memory is allocated to the object dynamically. Therefore add
markers whenever the \textit{new} keyword occurs in the program. The \textit{new} operation in Java invokes the constructor of the class, thus add the \textit{0xbee} marker in the constructor of the class whose memory usage need to be calculated. Insert this marker after every field of the object is accessed in the constructor. In case the class uses default constructor, write your own constructor which accesses all the fields of the class. Suppose there is a class \textit{Employee} and its constructor is:
\begin{verbatim}
public Employee(int id, String name, int age) {
    this.id = id;
    this.name = name;
    this.age = age;
    this.next = null;
}
\end{verbatim}
Annotate this constructor as follows, marker is a global volatile static integer variable.
\begin{verbatim}
public Employee{
    this.id = id;
    marker = 0xbee;
    this.name = name;
    marker = 0xbee;
    this.age = age;
    marker = 0xbee;
    this.next = null
    marker = 0xbee;
}
\end{verbatim}

\item Reference Updation\\
During the lifespan of the program, a reference may point to different objects. At somepoint of time during the execution of the program, there may be an object which is not
referred to by any reference. Then we can reclaim the memory allocated to that object,
thus we keep track of updates to references. In order to do this add the function call
\textit{dummy gc(Employee emp1, Employee emp2)} whenever there is a reference update i.e \textit{Employee emp1} starts pointing to \textit{Employee emp2} object. We decrement the reference count of the object to which emp1 was previously pointing to and increment the reference count of \textit{emp2}, to which \textit{emp1} now points to. The \textit{0xdea} marker is used for the object whose reference count has decremented and \textit{0xdee} marker is used for the object whose reference count has incremented. We use this information in the hardware for maintaining the reference count and when the reference count of the object becomes zero that memory is reclaimed. Suppose
there is a statement \textit{emp1.next = emp2	} in the program, insert the function call
\textit{dummy-gc()} before this statement. The annotated source code will be:

\begin{verbatim}
dummy_gc(Employee emp1, Employee emp2)
emp1.next = emp2
\end{verbatim}

Add \textit{dummy-gc()} as the \textit{Employee} class member function with following definition:
\begin{verbatim}
dummy_gc(Employee emp1, Employee emp2){
if(a != null){
  a.next.id = a.next.id;
  marker = 0xdea;
  a.next.name = a.next.name;
  marker = 0xdea;
  a.next.age = a.next.age;
  marker = 0xdea;
}
if(b != null){
  b.id = b.id;
  marker= 0xdee;
  b.name = b.name;
  marker = 0xdee;
  b.age = b.age;
  marker = 0xdee;
}
}
\end{verbatim}
\end{itemize}

\subsection{Collecting Traces}
Use Jikesrvm to get the traces of annotated benchmarks and disassemble the traces as described in above sections.

\section{Processing Traces}
\begin{verbatim}
$python postprocess (fin_trace.txt_0)
\end{verbatim}
\textit{fin\_trace.txt\_0} is the disassembled trace file.\\
The traces are post processed to mark the addresses of the objects which will be managed bythe hardware garbage collector. The hardware garbage collector maintains the reference count of the objects. When the object is created its reference count is one. When a new reference to the object is created the reference count is incremented by one. And when the reference is deleted the reference count is decremented by one. The memory space is dead when its reference count becomes zero. The markers in the annotated benchmarks are captured in the trace file and special instructions for allocation, incrementing and decrementing reference count are inserted in the trace files. Three type of instructions \textbf{allocate, ref\_inc and ref\_dec} are added to the trace file after extracting the useful information from the disassembled trace file.

\begin{table}[h]
\centering	
\begin{tabular}{|c|l|c|}
\hline
\multicolumn{1}{|l|}{\textbf{Marker in source code}} & \textbf{x86 instruction in trace file}              & \multicolumn{1}{l|}{\textbf{Instruction inserted in trace file}} \\ \hline
0xbee marker                                         & \multicolumn{1}{c|}{Push dword 0xbee or Push 0xbee} & allocate address                                                 \\ \hline
0xdee marker                                         & Push dword 0xdee or Push 0xdee                      & ref\_inc address                                                 \\ \hline
0xdea marker                                         & Push dword 0xdea or Push 0xdea                      & ref\_dec address                                                 \\ \hline
\end{tabular}
\caption{Instructions inserted in trace files for memory managment by hardware}
\end{table}

\subsection{Run on Tejas}
After postprocessing the traces, there are we inserted certain instructions in the trace file which are used by the hardware garbage collector in Tejas for memory management.
To calculate the memory usage with garbage collector set the Tejas config file parameters as follows:
\begin{enumerate}
\item \textit{memory usage = true}
\item \textit{garbage collection = true}
\end{enumerate}
To calculate the memory usage without garbage collector set the Tejas config file parameters as follows:
\begin{enumerate}
\item \textit{memory usage = true}
\item \textit{garbage collection = false}
\end{enumerate}

\end{document}